%%%%%%%%%%%%%%%%%%%%%%%%%%%%%%%%%%%%%%%%%
%% template_ppt.tex
%% Copyright 2017 Elias Alves @unixelias
%
% This work may be distributed and/or modified under the
% conditions of the LaTeX Project Public License, either version 1.3
% of this license or (at your option) any later version.
% The latest version of this license is in
%   http://www.latex-project.org/lppl.txt
% and version 1.3 or later is part of all distributions of LaTeX
% version 2005/12/01 or later.
%
% This work has the LPPL maintenance status `maintained'.
% 
% The Current Maintainer of this work is Elias Alves @unixelias
%
% This work consists of the files pig.dtx and the derived
% file beamerthemeAir.sty.
%
%%%%%%%%%%%%%%%%%%%%%%%%%%%%%%%%%%%%%%%%%%
%% Template para Apresentações tipo PPT
%% LaTeX Template
%% Version 1.0 (20/12/2017)
%%%%%%%%%%%%%%%%%%%%%%%%%%%%%%%%%%%%%%%%%%
%% Esse template teve como base:
%% https://pt.sharelatex.com/templates/presentations/air-beamer-template
%%%%%%%%%%%%%%%%%%%%%%%%%%%%%%%%%%%%%%%%%%
% Document License:
% CC BY-NC-SA 4.0 (http://creativecommons.org/licenses/by-nc-sa/4.0/)
%
% Para saber como usar o template "beamer" veja o link:
%%http://w3.ufsm.br/petmatematica/images/minicursos/Apostilas/apostila_minicurso_beamer.pdf

%%%%%%%%%%%%%%%%%%%%%%%%%%%%%%%%%%%%%%%%%
\documentclass[12pt]{beamer}
%% Para retirar seções
%%\documentclass[12pt,handout]{beamer}

\usepackage{beamerthemeAir}
\usepackage{thumbpdf}
\usepackage{wasysym}
\usepackage{ucs}
\usepackage[utf8]{inputenc}
\usepackage[brazilian]{babel}
\usepackage[T1]{fontenc} 
\usepackage{ae} 
\usepackage{amsfonts}
\usepackage{amssymb}
\usepackage{multicol}
\usepackage{pgf,pgfarrows,pgfnodes,pgfautomata,pgfheaps,pgfshade}
\usepackage{verbatim}
\usepackage{graphicx} % para inclusão de imagens
\usepackage{booktabs} % Allows the use of \toprule, \midrule and \bottomrule in tables

\usepackage{minted}
\definecolor{fundo-cod}{rgb}{0.95,0.95,0.95}
\definecolor{exemplo}{HTML}{EAF3FA}
% ---
% para tratar urls longas
\usepackage{url}				% para tratar URLs longas
\makeatletter
\g@addto@macro{\UrlBreaks}{\UrlOrds}
\makeatother

% ---
% Pacotes para inserir licença do documento
% ---
\usepackage[
	type={CC},
	modifier={by-nc-sa},
	version={4.0},
	]{doclicense}


\usepackage{wrapfig}
\usepackage{hyperref}
\usepackage{tikz}
\usepackage{movie15}
%\usepackage{multimedia}
%\usepackage{geometry}
%% Para inserir comentarios nos slides
%% Cria um slide na tela ao lado com as anotações
\usepackage{pgfpages}
%\setbeameroption{show notes}
%\setbeameroption{show notes on second screen=right}

% Imprimir slides para envio 4x1
%\pgfpagesuselayout{4 on 1}[a4paper,landscape]


\pdfinfo
{
	/Title		(Bancos de Dados II)
	/Subtitle		(Armazenamento)
	/Creator		(Departamento de Computação UFVJM - 2017)
	/Author		(Elias Alves)
	/Subject		(Disciplina especial 2017-4-esp)
}


\title[Abrev. Tít]{Título do Trabalho}
\subtitle{Subtítulo}
\author[Elias Alves]{Professor Elias Alves}
\institute{Universidade Federal dos Vales do Jequitinhonha e Mucurí \\
	\includegraphics[width=.2\textwidth,height=.3\textheight]{../img/main/brasao_ufvjm}%
}
\date{dezembro de 2017}

\begin{document}
	
	%%----------------------------------------------------------------------
	\begin{frame}
	\maketitle
	\end{frame}
%%----------------------------------------------------------------------
\AtBeginSection[]
{
	\frame<handout:0>
	{
		\frametitle{Sumário}
		\tableofcontents[currentsection=show,
		currentsubsection=show, 
		hideothersubsections, 
		sectionstyle=show,
		subsectionstyle=show,]
	}
}

\AtBeginSubsection[]
{
	\frame<handout:0>
	{
		\frametitle{Índice}
		\tableofcontents[currentsection=show/shaded,
		currentsubsection=show/shaded, 
		hideothersubsections, 
		sectionstyle=show/shaded,
		subsectionstyle=show/shaded,
		]
	}
}

\newcommand<>{\highlighton}[1]{%
	\alt#2{\structure{#1}}{{#1}}
}

\newcommand{\icon}[1]{\pgfimage[height=1em]{#1}}


%%%%%%%%%%%%%%%%%%%%%%%%%%%%%%%%%%%%%%%%%
%%%%%%%%%% Slides começam aqui %%%%%%%%%%
%%%%%%%%%%%%%%%%%%%%%%%%%%%%%%%%%%%%%%%%%

\section{Introduction}

\begin{frame}
  \frametitle{Prerequisites \& Goals}
  \framesubtitle{Knowledge is a brick wall that you raise line by line forever}
  \begin{block}{LaTeX}
  \begin{itemize}
    \item Obviously some basic LaTeX knowledge is necessary
    \item Some more features will be provided here
  \end{itemize}
  \end{block}

  \begin{block}{Beamer}
  \begin{itemize}
    \item You'll learn them by looking at this presentation source
  \end{itemize}
  \end{block}

  \begin{block}{Goal}
  \begin{itemize}
    \item Learn how to make well structured slides
    \item Using a beautiful theme (congrats to the Oxygen team!)
    \item Take over the world
    \item Relax...
  \end{itemize}
  \end{block}
\end{frame}

\section{Basic structuring}
\begin{frame}
  \frametitle{Sections, Frames and Blocks}
  \framesubtitle{Put everything into boxes}

  The current section is "Basic structuring". And the current frame
  is what you have on the screen right now.

  \begin{block}{A beautiful block}
  A block has a title, and some content. You can put in a block
  almost everything you want that is provided by LaTeX. For example
  math works as usual:
    \begin{equation}
    \sum_{i=1}^n i = \frac{n \times (n+1)}{2}
    \end{equation}
  \end{block}

  Also works outside a block:
  \begin{equation}
  \sum_{i=1}^n i^2 = \frac{n \times (n+1) \times (2n+1)}{6}
  \end{equation}
\end{frame}

\begin{frame}
  \frametitle{Different type of blocks}
  \framesubtitle{Weeeee! Colors!!}
  \begin{block}{Standard block}
  \begin{itemize}
    \item A standard block, used for grouping
    \item Obviously can contain itemizes too...
    \begin{itemize}
      \item And nested itemizes...
      \item of course!
    \end{itemize}
  \end{itemize}
  \end{block}
  \begin{alertblock}{Alert block}
  WARNING: Something very important inside this block!
  \end{alertblock}
  \begin{example}
  Note that examples are displayed as a special block...
  \end{example}
\end{frame}

\section{Fancy features}
\begin{frame}
  \frametitle{Highlighting}
  \framesubtitle{Hey! Look here! here!}

  \begin{block}{A regular block}
  \begin{itemize}
    \item Normal text
    \item \highlighton{Highlighted text} to draw attention
    \item \alert{"Alert'ed" text} to spot very important information
    \item Alternatively you can
    \begin{itemize}
      \alert{\item "Alert" the item itself}
      \highlighton{\item Or "Highlight" it}
    \end{itemize}
  \end{itemize}
  \end{block}
  \begin{alertblock}{If it's very very important...}
  \alert{... you can "alert" in an "alertblock"}\\
  Ewww, nasty, heh?
  \end{alertblock}
\end{frame}

\newcommand{\putlink}[1]{%
   \pgfsetlinewidth{1.4pt}%
   \pgfsetendarrow{\pgfarrowtriangle{4pt}}%
   \pgfline{\pgfxy(1,1)}{\pgfxy(#1,1)}
}

\begin{frame}
  \frametitle{Overlay effects}
  \framesubtitle{Keep the suspense!}
  \begin{block}{Time bomb}
  \begin{enumerate}
    \item<2-> Two more to go
    \item<3-> One more to go
    \item<4-> Last chance...
    \item<5-> BOOM!
  \end{enumerate}
  \end{block}
  \begin{block}{"Animation"}<6->
    \begin{pgfpicture}{0cm}{0cm}{7cm}{2cm}
    \only<1-6>{
      \putlink{2}
    }
    \only<7>{
      \putlink{4}
    }
    \only<8>{
      \putlink{6}
    }
    \only<9>{
      \putlink{8}
    }
    \only<10>{
      \putlink{10}
    }
    \end{pgfpicture}
  \end{block}
\end{frame}

\section*{}
\frame{
  \vfill
  \centering
  \highlighton{
  \usebeamerfont*{frametitle}And now?

  \usebeamerfont*{framesubtitle}Enter the secret section
  }
  \vfill
}
\begin{frame}
  \frametitle{Contributing to this beamer style}
  \framesubtitle{We want you !}

  \begin{block}{Why?}
  \begin{itemize}
    \item Beamer is hot!
    \item This style deserves to be improved
  \end{itemize}
  \end{block}

  \begin{block}{How?}
  \begin{itemize}
    \item Grab it
    \item Improve its LaTeX code
    \item Use you artistics skills
    \item Document it
    \item Help other people to use it
    \item Use it...
  \end{itemize}
  \end{block}
\end{frame}

\begin{frame}
  \frametitle{Resources}
  \framesubtitle{If you want to improve this style}
  \begin{thebibliography}{10}

  \beamertemplatearticlebibitems

  \bibitem{beamer-homepage}
    LaTeX Beamer
    \newblock {\tt http://latex-beamer.sourceforge.net/}

  \bibitem{kdeslides}
    KDE Presentations
    \newblock {\tt http://www.kde.org/kdeslides/}

  \end{thebibliography}
\end{frame}

\frame{
  \vspace{2cm}
  {\huge Questions ?}

  \vspace{3cm}
  \begin{flushright}
    Konqi Konqueror

    \structure{\footnotesize{konqi@kde.org}}
  \end{flushright}
}
\frame{
	\frametitle{Perguntas ?}
	\vspace*{\fill}
	{\huge Boas Festas!}  
	\vspace*{\fill}
	\begin{flushright}
		Elias C. Alves \\
		@unixelias \\
		% imprime a licença do trabalho
		\structure{\footnotesize{Decom/FACET - 2017}}
	\end{flushright}
	\begin{center}
		\doclicenseThis
	\end{center}
}
\end{document}
